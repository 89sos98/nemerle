\newcommand{\net}[0]{{\tt .NET}}
\newcommand{\kw}[1]{{\textcolor{kwcolor}{\tt #1}}}

\definecolor{kwcolor}{rgb}{0.2,0.4,0.0}
\definecolor{lgray}{rgb}{0.8,0.8,0.8}

\title{Nemerle}
\author{Micha{\l} Moskal}
\institute{Computer Science Institute, University of Wroc{\l}aw, Poland \\ 
Second ROTOR Workshop 2005, Redmond, WA}
\date{September 20, 2005}


\begin{document}

\section{Introduction}

\frame{\titlepage}

\frame{
\frametitle{What's that?}
\begin{itemize}
  \item high--level, statically typed programming language
    \begin{itemize}
      \item object--oriented, functional, imperative
      \item type inference
      \item pattern matching
      \item Turing-complete macros -- compiler plugins extending the language
     \end{itemize}
  \item since the beginning designed for the \net
\end{itemize}
}

\section{Reasons}


\frame{
\frametitle{Why \net\ ?}

\begin{itemize}
  \item wide variety of libraries
  \item multi--language paradigm
  \item runtime environment (GC, JIT)
  \item portable executable files (Microsoft \net, Mono, DotGNU, Rotor)
  \item dynamic class loading and code generation
\end{itemize}
}

\frame{
\frametitle{Why a new language, from a user perspective?}

\begin{itemize}
  \item combining object--oriented and functional programming
  \begin{itemize}
    \item familiar, C\#--like, object--oriented top--level program structure
          (classes, interfaces)
    \item methods can be implemented in a functional style
    \item easy access to imperative features
    \item start: looks like ML, has a subset of C\# features + functional stuff
    \item end: looks like C\#, includes C\# + functional stuff + ...
  \end{itemize}
  \item macros!
\end{itemize}
}

\frame{
\frametitle{Why a new language, from a computer scientist perspective?}
\begin{itemize}
  \item experimentation
  \begin{itemize}
    \item meta--programming system
    \item type inference algorithms
    \item programming language design
    \item .NET code generation using S.R.E. API
    \item stress testing generics
  \end{itemize}
\end{itemize}
}

\frame{
\frametitle{And now something completely different:}

\begin{center}
\Large Sample time!
\end{center}
}


\section{Macros}
\frame{
\frametitle{Macros}
\begin{itemize}
  \item written in Nemerle
  \item dynamically loaded compiler modules (no connection with CPP!)
  \item transform, generate and analyse programs
  \item can extend syntax of the language
  \item can interact with type inference
  \item work on syntax trees of expressions and types
  \item can read external files, query database etc.
\end{itemize}
}


\frame{
\frametitle{Uses of macros}
\begin{itemize}
  \item specialized sublanguages ({\tt printf}, {\tt scanf}, regular expressions,
    SQL, XML, XPath)
  \item generating trees from other trees (serialization, specialization of code)
  \item domain specific optimization
  \item assertion system
  \item automatic creation of repeatable class level patterns  
  \item \textit{Aspect Oriented Programming}
\end{itemize}
}

\frame[containsverbatim]{
\frametitle{Example use of macro}
This macro checks syntax and type validity of a query at compile-time
(by connecting to a database). It creates code, which uses {\tt SqlParameter}
to pass value of {\it myval} to {\tt SqlCommand} securely.

\begin{verbatim}
def myval = "Kate";
sqlloop ("SELECT * FROM employee WHERE"
         " firstname = $myval", dbcon) 
{
  printf ("%s %s\n", firstname, lastname)
}
\end{verbatim}
}


\section{The project}

\frame{
\frametitle{Compiler}

\begin{itemize}
  \item bootstrapping
  \item generates and consumes generics
  \item 0.9.0 release, for August CTP and Mono 1.1.9
\end{itemize}
}


\frame{
\frametitle{Projects using Nemerle}

\begin{itemize}
  \item Sioux -- HTTP/application server (founded from the grant)
  \item cs2n -- C\# to Nemerle converter (founded from the grant)
  \item nemish -- Nemerle Interactive Shell
  \item Asper IDE/editor
  \item RiDL parsing/lexing tools
  \item NAnt build system plugin
  \item CodeDom generator (ASP.NET support)
  \item Code Completion Engine
  \item IDE integration (VS.NET, \#D, MD)
\end{itemize}
}


\frame{
\frametitle{Impact and community}

\begin{itemize}
  \item .NET runtime issues:
  \begin{itemize}
     \item several SRE issues reported to MSDN Product Feedback and
           one serious issue with static field initialization in generic classes
     \item even more issues reported to the Mono team (the runtime is much less
           mature)
  \end{itemize}
  \item mailing list (70 subscribers)
  \item web forum (recently started)
  \item issue tracker -- 500 issues total, 50 still open
  \item wiki--based webpage -- external documentation writers/fixers
  \item online course will start around October 1st
  \item completed regular course at our Institute
\end{itemize}
}


\frame{
\frametitle{Performance results}

\begin{itemize}
  \item interested in performace since compiler bootstrap
  \item tail calls
  \begin{itemize}
    \item the {\tt tail.} prefix makes the call go much slower
    \item use jumps where possible
    \item make {\tt tail.} generation default off
  \end{itemize}
  \item delegates vs functional objects -- delegates are much slower
  \item boxing vs generics -- generics seems faster, but hard to compare
\end{itemize}
}


\frame{
\frametitle{TODO}

\begin{itemize}
  \item 1.0 stable release approaching
  \item more IDE integration
  \item more community building
  \item more static analisis
\end{itemize}
}

\section{The end}
\frame{
\frametitle{More info}
\begin{center}
\Large \textcolor{blue}{\tt http://nemerle.org/} \\
\vspace*{2cm}
Questions?
\end{center}
}
\frame{
\frametitle{Thank you!}
\begin{center}
\Large Thank you for your attention!
\end{center}
}
\end{document}

% vim: language=english
