\documentclass{acm_proc_article-sp}
\usepackage[latin2]{inputenc}
\usepackage{fancyvrb}
\usepackage{fancyhdr}
\usepackage{hcolor}
\usepackage{color}
\usepackage[a4paper, top=2.5cm, left=2.5cm, right=2.5cm, bottom=3.5cm]{geometry}

\textheight 9.33in
\textwidth 6.3in         % Width of text line.
                        % For two-column mode:
\columnsep 0.8cm        %    Space between columns
\columnseprule 0pt      %    Width of rule between columns.
\hfuzz 1pt              % Allow some variation in column width, otherwise it's
                        % too hard to typeset in narrow columns.

\hyphenation{}

\setcounter{secnumdepth}{1}

\begin{document}

\balancecolumns

%\conferenceinfo{WOODSTOCK}{'97 El Paso, Texas USA}
%\setpagenumber{50}
%\CopyrightYear{2002} % Allows default copyright year (2002) to be over-ridden - IF NEED BE.
%\crdata{0-12345-67-8/90/01}  % Allows default copyright data (X-XXXXX-XX-X/XX/XX) to be over-ridden.

\permission{}

\newcommand{\net}[0]{{\tt .NET}}
\newcommand{\netf}[0]{{\tt .NET} framework}
\newcommand{\nem}[0]{Nemerle}
\newcommand{\cs}[0]{C\#}
\newcommand{\kw}[1]{{\tt \bf #1}}

\title{Nemerle}
\subtitle{\vspace*{-3mm}Introduction to a Functional .NET Language}

\DefineVerbatimEnvironment
  {Code}{Verbatim}
  {xleftmargin=0mm,%
   xrightmargin=0mm,framesep=2mm}

\numberofauthors{3}
\author{
\alignauthor Micha� Moskal\\
       \affaddr{University of Wroc�aw}\\
       \affaddr{Computer Science Institute}\\
       \affaddr{Przesmyckiego 20}\\
       \affaddr{Poland, 50-151 Wroc�aw}\\
       \email{moskal@nemerle.org}
\alignauthor Pawe� W. Olszta\\
       \affaddr{University of Wroc�aw}\\
       \affaddr{Computer Science Institute}\\
       \affaddr{Przesmyckiego 20}\\
       \affaddr{Poland, 50-151 Wroc�aw}\\
       \email{olszta@nemerle.org}
\alignauthor Kamil Skalski\\
       \affaddr{University of Wroc�aw}\\
       \affaddr{Computer Science Institute}\\
       \affaddr{Przesmyckiego 20}\\
       \affaddr{Poland, 50-151 Wroc�aw}\\
       \email{skalski@nemerle.org} \\
}
\date{\today}
\global\hack{
\parbox[t]{6.3in}{
  \vspace*{-5mm}
  \begin{center}
  \secfnt ABSTRACT
  \end{center}

  \nem\ is a new functional language designed from the ground up for the \net.
  In this paper we have focused on features absent in traditional ML-like and object-oriented 
  languages:
  variant inheritance, assertions and powerful code-generating macros. We also 
  gave concern for the syntax and the ``spirit'' of \nem\ that makes it a good
  transition language for programmers with \cs\ background.
  
  \vspace*{-1mm}
  \keywords{Functional programming, programming languages, metaprogramming.}
  \vspace*{4mm}
}
}

\maketitle

\section{Motivation}

Our objective was to create a statically typed functional language with well 
founded \net~\cite{CLI}\ interoperability. The \net\ environment, especially 
since the introduction of generics~\cite{Generics}, provides an excellent 
platform for high-level language implementation which:
\begin{itemize}
  \item comes with a rich class library in the core system
  \item gives access to vast cache of additional third party libraries
  \item provides automatic garbage collection and security features
  \item handles native code generation and low-level optimizations (JIT)
  \item guarantees portability of executables
  \item allows integration with existing development tools
  \item etc. etc.
\end{itemize}
Of course, the framework is strongly object-orie\-nted and primarily focused on traditional
object-orie\-nted and imperative languages. Therefore po\-rts of the existing functional 
languages to the \net\ did not fit in as well as, for example, \cs~\cite{CS}
does. Addressing this issue was the main idea behind the design of \nem.

In comparison to Haskell~\cite{Haskell} or SML~\cite{SML}, \nem\ is not a pure 
language in the functional sense, allowing the programmer to create completely 
object-oriented and imperative programs. This makes \nem\ a good transition language for 
people with C-like imperative and object-orie\-nted languages background. They can take 
advantage of the language imperative features until they gradually learn 
how to program in a functional fashion.

An easy access to imperative constructs is only one of the requirements needed 
in such a transition language. Probably the hardest thing about learning ML 
is understanding the compiler error messages about typing mismatches. It
may seem odd at first glance, but this is the reality -- the type inference
is very nice when it works, but when it fails, you are stuck with error
messages hundred lines from the place of the the real error. 

We have decided to avoid language constructions that produce typing errors in ML, 
while generating syntax errors in other languages (for example function application 
being just $\epsilon$); requiring the typing to be explicit, at least for global 
functions -- implicit typing is not really possible to achieve when aiming for
a good support for methods overloading.

It seems easy to observe that it is the quality of the design of the
object-oriented system that determines usability of a programming
language.  While the existing object-oriented extensions to functional
languages are appealing because of their elegance, they do not fit the
\netf\ at all. We have decided to make our object-oriented system simply
mirror the \net\ design.


\section{Overview}

At the high level \nem\ can be characterized as a combination of \cs\
at the class level and a ML-like language at the expression level. 
However, the syntax of the ML fragment is much less ambiguous and 
more C-like than Algol-like. The result is an expression-oriented 
language with a feeling of \cs.

Of course we also need variants\footnote{Called datatypes in SML,
and sometimes sum types in Caml.}, pattern matching and functional
values. These can be thought of as extensions to the base 
\cs-like language.

There are some other facts about \nem\ that are implied by the 
``not-so-ML-like \net\ language'' paradigm:
\begin{itemize}
  \item The language is statically typed, but dy\-na\-mic casting is available
        and can be used when nee\-ded.
  \item The language combines functional, object-oriented, and imperative features.
  \item The object system is a one-to-one mapping of CLR's -- making it
        fairly easy to understand.
  \item The language interoperates fully with other \net\ languages -- it
        is both a CLS consumer and producer.
\end{itemize}
In the following sections we will show how the language looks like and
how is it different from ML and \cs. The reader is assumed to have some
basic knowledge about both ML and \cs.

It is important to mention that the language is still evolving and that
its design is quite flexible. Especially assertions and macros are 
relatively new features. We are open to any suggestions.


\section{The language}

The top-level program structure reassembles \cs. Th\-ere are namespaces,
then classes and finally methods. We also have modules (classes with all
members static and public) and variants. Let us look at the famous example:

\begin{Code}
  class Hello {
    public static Main () : void {
      System.Console.WriteLine ("Hello, " 
        + "I have {0} years!", 22);
    }
  }
\end{Code}

Another way to write it could be:

\begin{Code}
  using System.Console;

  module Hello {
    public Main () : void {
      WriteLine ("Hello cruel world.");
    }
  }
\end{Code}

The basic building block of a method is a sequence. A sequence groups local
definitions (specified with the \kw{def} keyword), expressions computed for
their side effects and the final expression returned as the value of entire
sequence\footnote{We put here value bindings and side-effect expressions
into one can. This is exactly how it works in imperative languages and 
(under the hood) in eager functional languages. It models real world
behavior better, and should be easier to understand.}.

\begin{Code}
public static factorial (x : int) : int 
{
  def loop (acc, x) {
    if (x <= 1)
      acc
    else
      loop (acc * x, x - 1)
  };
  loop (1, x)
}
\end{Code}

In this example the local function is implicitly typed -- its type is
inferred automatically by the compiler. Global functions are explicitly
typed by design of the language.


\subsection{Mutable values}

Mutable local values are defined using declarations like 
\kw{mutable}\ $x$\ {\tt <-}\ $expression${\tt ;}. The value $x$ can be 
later used as a value bound with \kw{def} without any explicit dereference 
operator\footnote{Like the {\tt !} operator in ML.}, but can be 
assigned using the assignment operator ({\tt <-}).

\begin{Code}
public static factorial (x : int) : int 
{
  mutable acc <- 1;
  mutable k <- n;
  while (k > 0) {
    acc <- acc * k;
    k <- k - 1;
  };
  acc
}
\end{Code}

The \kw{while} loop should be considered as just a different form of tail
recursion. It is in fact implemented as a macro which generates the
following code:

\begin{Code}
public static factorial (x : int) : int 
{
  mutable acc <- 1;
  mutable k <- n;
  def loop () {
    when (k > 0) {
      acc <- acc * k;
      k <- k - 1;
      loop ()
    }
  };
  loop ();
  acc
}
\end{Code}

Our optimizer is clever enough to recognize that it needs no new \verb,loop,\
method here -- it will just insert the \verb,br,\ opcode at the IL level.

The \kw{mutable} keyword can be also used as a modifier on fields. The contents
of such fields can be modified using the same assignment operator.


\subsection{Variants and pattern matching}

Variants are compiled to subclassing and should be thought of as subtypes.
For example:

\begin{Code}
variant BinaryTree <'a> {
  | Leaf { val : 'a; }
  | Node { left : BinaryTree <'a>; 
	   val : 'a; 
	   right : BinaryTree <'a>; }
} 
\end{Code}

Would be compiled to:

\begin{Code}
class BinaryTree<A> {}
class Node<A> : BinaryTree<A> { 
  BinaryTree<A> left; 
  A val; 
  BinaryTree<A> right; 
}
class Leaf<A> : BinaryTree<A> {}
\end{Code}

However, in the absence of of generics support in the current Framework release
type qualifiers are stor\-ed as attributes alongside the type declarations.

Of course we can use regular ML-like matching over variants:

\begin{Code}
count<'a> (t : BinaryTree <'a>) : int {
  match (t) {
    | Node (l, _, r) => 
      count (l) + 1 + count (r)
    | Leaf => 1
  }
}
\end{Code}

Which can be shortened to:

\begin{Code}
count<'a> (t : BinaryTree <'a>) : int {
  | Node (l, _, r) => 
    count (l) + 1 + count (r)
  | Leaf => 1
}
\end{Code}

The \verb|'a|\ after \verb|count|\ quantifies following occurrenc\-es of
\verb|'a|.
                                                                                                              
There is one tricky thing about the second line of our example. It could have
been written in any of the following ways:

\begin{Code}
  | (Node) as n => 
    count (n.left) + 1 + count (n.right)
  | Node (l, _, r) => 
    count (l) + 1 + count (r)
  | Node { left = l; right = r } => 
    count (l) + 1 + count (r)
  | Node { left = l; val = _; 
           right = r } =>
    count (l) + 1 + count (r)
\end{Code}

In fact, when the compiler sees a tuple pattern and expects a record pattern,
the tuple is transformed into a record. It is therefore not so painful
to require variant members to be named.

It is also possible to have deep patterns like Foo (Bar (Baz)),
to match constants and to match real tuples. We also support pattern 
guards -- that is condition checked after pattern has matched.


\subsection{Variant inheritance}

The subtyping model allows the variants to carry slightly more
information then their ML counterparts. In particular it is possible to
make the variant base class have some fields, methods or even derive
from some other class. This way all variant options can have some
common part. An example (taken from \nem\ compiler, which is written in
\nem\ itself):

\begin{Code}
class Located {
  file : string;
  line : int;
}

variant Expr extends Located {
  | E_call { fn : Expr; 
             parms : list <Expr>; }
  | E_ref { name : string; }
}

public static dump (e : Expr) : void {
  print ("// " + e.file + ": " + 
         e.line.ToString ());
  match (e) {
    | E_ref (name) => print (name)
    | E_call (fn, parms) =>
      dump (fn);
      List.iter (dump, parms)
  }
}
\end{Code}


\subsection{Constrained parametric types}

Types can be parametrized over other types. Type arguments can be
constrained. This works the same way as generics do in IL. It is also 
possible to parametrize methods.

\begin{Code}
variant tree <'a> 
where 'a : IComparable <'a> 
{
  | Node { 
      left : tree <'a>; 
      data : 'a; 
      right : tree <'a>; 
    }
  | Tip
}
\end{Code}

This is the \nem\ way to do things that would have been done with
functors in ML-like langu\-a\-ges. It is not strictly as powerful, but 
seems to be good enough in practice and integrates well with 
the \netf.


\section{Assertions}

Currently we have C-like \kw{assert}\ implemented as a macro. We plan implement
have \kw{require}\ and \kw{ensure}\ to support design by contract, as well
as several special assertions for mutable value enforcing invariants.
\begin{itemize}
  \item mutable values \kw{guarded}\ with assertions -- update of this very 
        value triggers associated assertion
  \item \kw{guard} assertions that are checked after update of any value 
        directly referenced from the assertion body; checks are performed 
        until the end of the current block
\end{itemize}
It is possible to attach the \kw{guard}\ assertions to local values, instance
fields and static fields (global values).

We sometimes want assertions like \verb,x + y == 5,\ to hold, with mutable
\verb,x,\ and \verb,y,. To allow update of \verb,x,\ immediately followed
by update of \verb,y,\ a \kw{transaction}\ block is introduced. Assertions to
be triggered during the \kw{transaction}\ block are stacked, and executed
when the control leaves it.

It is to be reconsidered when exactly assertions are checked. Enforcing
a check after each update can be hard in presence of parameters passed
by \kw{ref}.


\section{Macros}

Macros in \nem\ have much more to do with Meta Haskell~\cite{MetaHaskell}, 
CamlP4~\cite{CamlP4} or Sche\-me Lisp code-generating macros, than with macros 
in the languages like C. Macros are essentially compiler plugins -- pieces 
of the \nem\ code that take type or expression abstract syntax trees (or AST for short) 
and return some other expressions or types (also as AST).

Macros are by definition Turing-complete\footnote{It is not by accident
like in some other languages.}. Mac\-ros can access external files,
extract typing information from a running database and generally
do whatever you can imagine.

Macros are executed at the compilation time. The code they generate is later
statically type checked. Macros are thus safe. There is always a risk that
a macro will crash (or loop) during the compilation, but there is no way
to avoid that while retaining its expressiveness.

As said before, the macros are written in Nemerle itself. In principle it 
would be possible to use any other \net\ language, but \nem\ provides a special
code quotation syntax to construct and walk its own AST. It provides a clear
separation of the meta-language from the object language it is describing.

The macros can be also executed at the run time, taking advantage of dynamic
aspects of the \texttt{.NET} fra\-me\-work. This can be used for example to develop programming
language interpreters, or to specialize the code for efficiency.

Our meta-system is closely interleaved with the compilation process. 
It can perform partial typing of program's AST. Compiler
internal typing procedures are executed by macro code in arbitrary order
and their result can be analyzed, giving much more information about the program.

\subsection{Usage}

Example uses of macros:
\begin{itemize}
  \item extending the syntax of the language
  \item embedding special purpose sublanguages in \nem:
  \begin{itemize}
    \item \verb,printf,\ and \verb,scanf,\ like functions
    \item binding optional and named groups in regular expression to local 
          variables
    \item \verb,$,-interpolation like in Bourne shell or Perl   %$
    \item binding results of SQL queries to local variables in a type safe way
    \item special syntax for XPath or some other XML-matching constructions
  \end{itemize}

  \item generation of AST from external files
  \begin{itemize}
    \item Yacc and Burg-like tools
    \item generating types from an XML schema or DTD
  \end{itemize}

  \item generation of external files based on AST
  \begin{itemize}
    \item pretty printing of the generated or original co\-de
  \end{itemize}

  \item generation of AST based on other AST
  \begin{itemize}
    \item generating XML serialization methods
    \item specialization of the code at the source language level
    \item support for Aspects-Oriented Programming by adding 
          cross-cutting ``concerns'' to the program in algorithmic
          and arbitrarily flexible way   
  \end{itemize}
\end{itemize}
\subsection{Example: regular expression macro}

This macro matches given string against pattern in sequence
binding matched groups to variables. Not the use of \verb,printf,\
macro in this example.

\begin{Code}
regexp match (s) {
  | "a*.+" => printf ("a\n");
  | @"(?<num : int>\d+)-\w+" => 
    printf ("%d\n", num + 3);
  | "(?<name>(Ala|Kasia))? ma kota" =>
    match (name) {
      | None => printf ("noname?\n")
      | Some (n) => printf ("%s\n", n)
    }
  | _ => printf ("default\n");
}
\end{Code}

\subsection{Example: SQL queries macro}

This macro requires an SQL parser, and access to the database we are working on,
so that the types of table columns and stored functions can be determined. 
It is necessary to determine the types of SQL expressions, which can be later used
to produce source language bindings for values returned by SQL queries.

\begin{Code}
sql_loop (conn, 
  "SELECT salary, LOWER (name) AS lname"
  "  FROM employees"
  "  WHERE salary > $(min_salary * 3)")
    sql_print ("$lname : $salary\n")
\end{Code}

And the result:

\begin{Code}
def cmd = SqlCommand (
  "SELECT salary, LOWER (name)"
  "  FROM employees"
  "  WHERE salary > @parm1", conn);
cmd.Parameters.Add ("@parm1", 
                    min_salary * 3);
def r = cmd.ExecuteReader ();
while (r.Read ()) {
  def salary = r.GetInt32 (0);
  def lname = r.GetString (1);
  printf ("%s: %d\n", lname, salary)
}
\end{Code}


\subsection{Example: A sample macro implementation}

This is a sample implementation of a macro that adds the \cs -like
\verb,foreach,\ loop to the language (together with the special syntax
for this con\-str\-uct). 
This code comes directly from the compiler implementation.

\begin{Code}
macro @foreach (iter : funparm, 
                collection, body) 
syntax ("foreach", "(", iter, "in", 
        collection, ")", body)
{
  match (iter) {
    | <[ funparm: $(iname : var) 
                  : $ty ]> =>
      <[ def enumerator = 
           $collection.GetEnumerator ();
         while (enumerator.NextMove ()) 
         {
           mutable $(iname : var) <- 
             (enumerator.Current :> $ty);
           $body;
         }
      ]>        
    | _ =>
      Message.fatal_error (
        "iterator in `foreach' must be "
        + "id with optional type")
  }
}
\end{Code}

The code generated by the presented macro is constructed by 
the quotation construct in lines 6--13. Note that it creates code,
which uses another syntax-extending macro, the \verb,while,\ loop.

\section{Code generation}

The typed abstract syntax tree of expressions is converted into 
an intermediate functional description of a stack machine which
is later used to build the compiler output using the API of\\
\verb,System.Reflection.Emit,

Optimizations are performed on both the typed AST level as well as
on the intermediate representation level. For example tail calls
are marked as such during AST generation, while matching automata
generation is performed after intermediate code is generated.


\subsection{Tail call elimination}

We have implemented tail calls using the \verb,tail., prefix available in IL.
However, it did not bring any improvements to the execution speed, it even slowed
things down by a factor of 15\%.

For tail calls to the current function, we have implemented simple
transformation to argument assignment and \verb,goto,. It brought a little speed improvement (over
the version without tail calls), but reduced memory usage by about 20\% (compared to the same version). 
Later we have implemented real loops (that is we do not always generate new method for local
functions now), it made things faster by about 12\%.

\subsection{Matching optimizations}

We are working on good matching code generation using a hashing function
and binary search automates. This is, however, in a very early stage yet.


\section{Summary}

We have shown the key points of a new functional language for the \netf.
The language combines well-known concepts in a unique fashion. We believe 
that it could be used to teach the basics of functional programming and 
the \net. We also hope it can be used outside academia as a real life, 
industry language.

\begin{thebibliography}{9}
\bibitem[Kenn01]{Generics}
Kennedy A., Syme D. 
Design and implementation of generics for the .NET Common language runtime 
in ACM SIGPLAN 2001 conf. proc., Snowbird, Utah, ACM Press, pp. 1--12, 2001.

\bibitem[Mil91]{SML}
Milner R., Tofte M., Harper R. The Definition of Standard ML. The MIT Press, 1991.

\bibitem[Jon99]{Haskell}
Jones S. P., Hughes J. 
Report on the Programming Language Haskell 98: 
A non-strict, purely functional language.
Technical Report YaleU/DCS/RR-1106, Dept. of Computer Science, Yale University, 1999.

\bibitem[Shea02]{MetaHaskell}
Sheard T., Jones S. P.. Template meta-programming for Haskell. 
In Proceedings of the Haskell workshop, pp. 1--16.
ACM Press, 2002.

\bibitem[ISO03a]{CS}
International Organization for Standardization. C\# Language Specification,
ISO/IEC 23270:2003, 2003.

\bibitem[ISO03b]{CLI}
International Organization for Standardization. 
Common Language Infrastructure, ISO/IEC 23271:2003, 2003.

\bibitem[CamlP4]{CamlP4}
\texttt{http://caml.inria.fr/camlp4/}

\end{thebibliography}

\end{document}
