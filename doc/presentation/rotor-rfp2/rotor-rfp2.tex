\documentclass[a4paper,11pt]{article}
\usepackage[compat2,nofoot,headheight=14pt]{geometry}
\usepackage{fancyhdr}
\renewcommand{\ttdefault}{cmtt}
\renewcommand{\sfdefault}{pag}
\usepackage{url,array}

%\setlength{\parindent}{0pt}
%\setlength{\parskip}{5pt}

\overfullrule1pt

\hyphenation{De-bug-ger In-ter-pre-ter Spe-ci-fi-ca-tion}

\pagestyle{fancy}
\lhead{Nemerle project}
\rhead{\thepage}
\cfoot{}

\title{Nemerle project}
\author{Response to\\Second Rotor Request for Proposal}
\date{}

\begin{document}

\maketitle
\thispagestyle{empty}

\section{Overview}

\subsection{Project Title}

Nemerle.

\subsection{Contributor Names}
Micha{\l} Moskal, Pawe{\l} Olszta, Kamil Skalski, Prof. Leszek Pacholski, Marcin M{\l}otkowski
Tomasz Wierzbicki.

\subsection{Abstract}

The objective of this project is to design and implement a new hybrid
(functional, object-oriented and imperative) programming language for the
.NET platform. Key features of the language include: simple, similar to
C\# syntax, easy to use object system (derived directly from the .NET),
powerful code-generating macros, variants and pattern matching, type
inference. We plan to make the language full CLS consumer and producer,
which fits .NET as good as C\# does.

Investigating optimization oportunities is also very important, we
plan (tutaj tralala, jak to bedziemy patrzec w rotora, zeby lepiej
optymalizowac).

\section{Contact Details}


\subsection{Institution Details}

\begin{quote}
Institute of Computer Science\\
University of Wroc\l aw\\
ul.\ Przesmyckiego 20\\
PL--51--151 Wroc\l aw\\
Poland\\[2ex]
phone: 0048 71 3251271\\
fax:   0048 71 3756244\\
URL: \url{http://www.ii.uni.wroc.pl/}\\
URL: \url{http://nemerle.org/}
\end{quote}


\subsection{Contributor Details}

Micha{\l} Moskal (\texttt{michal.moskal@nemerle.org}), is master
student, and coordinator of the project. He was working on several large
programming projects, including few programming languages. URL: 
\url{http://www.kernel.pl/~malekith/}

Pawe{\l} Olszta (\texttt{pawel.olszta@nemerle.org}), is master student
and compiler developer. He has large industry experience, including
3 months spent in Redmond. 

Kamil Skalski (\texttt{kamil.skalski@nemerle.org}), is master student.
He is responsible for the macro design.

Prof. Leszek Pacholski (\texttt{pacholsk@ii.uni.wroc.pl}), director of The Institute.

Marcin M\l otkowski (\texttt{marcinm@ii.uni.wroc.pl}), lecturer at Wroc\l aw University.

Tomasz Wierzbicki (\texttt{tomasz@ii.uni.wroc.pl}), assistant at Wroc\l aw University.

There are also several other students involved in the project.


\section{Project Description}
\subsection{Background}
Insert pertinent information that describes the general background of
your research area that is relevant to this proposal. Use the References
section (below) where necessary.


\subsection{Project Details}

Describe your proposal in detail, specifying the project goals and how
these will be achieved. Discuss any requirements your project makes on
other work/technologies, along with any foreseeable limitations. You
may list URL's to websites where additional information can be found.


\subsection{Academic Relevance}

Briefly describe how your proposal extends, complements or varies from
previous work in this area. Discuss any opportunity for collaboration
with groups outside your institution.


\subsection{Experience}

If necessary, insert information about your group's or members' previous
research in fields relevant to this proposal. Point to entries in the
References section where necessary.

\section{Project Plan}

\subsection{Deliverables/Milestones}

\begin{itemize}
\item version 0.1, scheduled Feb 15 2004. Features:
  mostly stable compiler and language;
  preliminary version of \textit{Reference Manual}; 
  macros working at the expression level.

\item version 0.2, scheduled Apr 15 2004. Features:
  most of CLS compatibility;
  macros on type definitions;
  documentation generation system;
  good pattern matching optimizations;
  some other simple optmizations;
  production version of \textit{Reference Manual};
  preliminary version of \textit{Nemerle for C\# Programmers} tutorial.

\item version 0.3, scheduled Jun 15 2004. Features:
  full CLS compatibility (producer/consumer);
  more compiler optmizations;
  production version of tutorial.
\end{itemize}

\subsection{Intellectual Property}

We are going to publish research papers in the area, however main outcome
is language implementation and documentation. Both will be released
under BSD-style license. See \url{http://nemerle.org/license.html}
for details.


\section{Supporting Information}


\subsection{Costing}

Jakies konferencje

Przelot do Redmond

I sobie tez chcemy zaplacic


\subsection{References}

Gdzies trzeba sie odwolac do intro.pdf.


\subsection{Appendixes and/or Attachments}

Provide any appendixes or attachments, such as ``letters of support''
from your institution or external groups.

\end{document}
