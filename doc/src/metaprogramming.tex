\documentclass[draft,11pt]{article}
\usepackage[latin2]{inputenc}
\usepackage{fancyvrb}
\usepackage{fancyhdr}
\usepackage{hcolor}
\usepackage{color}
\usepackage[a4paper, margin=3cm]{geometry}

\newcommand{\net}[0]{{\tt .NET}}
\newcommand{\netf}[0]{{\tt .NET} framework}
\newcommand{\nem}[0]{Nemerle}
\newcommand{\cs}[0]{C\#}
\newcommand{\oo}[0]{object-oriented}
\newcommand{\kw}[1]{{\tt \bf #1}}

\DefineVerbatimEnvironment
  {Code}{Verbatim}
  {frame=lines,numbers=left,xleftmargin=15mm,%
   xrightmargin=10mm,framesep=2mm,framerule=1mm,%
   rulecolor=\color[rgb]{0.8,0.8,0.8}}

\begin{document}

\title{{\Huge \sc Meta-programming} \\ 
  In new Functional \net\ Language Nemerle \\
  {\large \tt http://nemerle.org/}}
\author{Kamil Skalski \\ {\small University of Wroc�aw}}
\date{\today}

\maketitle

\thispagestyle{empty}

\begin{abstract}
  We present design of meta-programming system embedded in new functional
  language \nem\ incorporated into \net\ platform. It gives wide set of
  operations performed on code at compile time, including program generation,
  transformation and automated analysis. It provides good support for Aspects 
  Oriented Programming, extending code with various features parametrized by
  external data, while still strong typying it during compilation.

  \nem\ and its macros are intended to easily introduce functional methods
  and power of meta-programming to programmers with C\#-like background. 
  Our focus is set on creating simple syntax, while not loosing power of
  its expresiveness.
\end{abstract}

\pagestyle{fancy}
\lhead{Kamil Skalski -- \it Nemerle}
\rhead{\thepage}
\cfoot{}


\section{General concept}

Idea of compile-time meta-programming has been studied for quite a long time.
It was incorporated into several languages, like Common-Lisp templates, C 
preprocessor-based macros, C++ template system and finally Haskell Template 
Meta-programming. They vary on their power and ease of use, but generally 
involves computations made during compile-time and generating code from
some definitions.

Our system is mainly designed for operating on parts of programs at compile-time,
but using features of /net/ and its dynamic code loading abilities, it is also
possible to run macros during run-time. 

We add some new ideas about syntax of meta language and usage of new technologies 
to connect great abilities of meta-programming with top industrial standards in
computer technology:
\begin{itemize}
  \item We develop uniform and simple quasi-quotation system, which doesn't require
    learning of any keywords to write quite complicated macros
  \item Using macros is transparent from user's point of view - it is not possible
    to distinguish between calling meta-program and common function, so user
    can use other's work without even knowing idea of meta-programming
  \item Syntax extensions allow even greater embedding of macros into language,
    which provides method to easily customize it to one's needs, without interfering
    with compiler internals
  \item Our system can be used to transform or generate practically any fragment
    of program, which composed with /net/ OO structure gives powerful tools for
    software engeneering methodologies
  \item We allow macros to type fragments of code, which they operate on, during
    their execution. It can be very useful to make them more independent of 
    shape of data they operate on
\end{itemize}

We combine results of previous works with our accomplishments to create consistent 
system of:
\begin{itemize}
  \item 
\end{itemize}

\section{Details}

\section{Conclusion}

\end{document}
