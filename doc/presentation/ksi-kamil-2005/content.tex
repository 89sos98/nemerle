\newcommand{\net}[0]{{\tt .NET}}
\newcommand{\kw}[1]{{\textcolor{kwcolor}{\tt #1}}}
\newcommand{\ra}{\texttt{ -> }}

\definecolor{kwcolor}{rgb}{0.2,0.4,0.0}
\definecolor{lgray}{rgb}{0.8,0.8,0.8}

\title{Nemerle}
\author{Kamil Skalski}
\institute{Instytut Informatyki Uniwersytetu Wrocławskiego \\
Spotkanie Koła Studentów Informatyki}
\date{17 listopada 2005}


\begin{document}

\section{Abstrakcja i uogólnianie w programowaniu}

\frame{\titlepage}

\frame{
\frametitle{Abstrakcja}
\begin{itemize}
  \item najlepsze są proste i bezpośrednie rozwiązania
  \item jednak wtedy niskopoziomowe detale wchodzą wszędzie gdzie tylko się da
  \item potrzeba podzielić problem na warstwy
  \item każda warstwa realizowana jest niezależnie
  \item kolejna warstwa jest abstrakcją poprzedniej
  \item ostatnia zawsze sprowadza się do hasła ``chcę mieć program, który działa'' 
    \footnote {tudzież, który uda mi się sprzedać}
\end{itemize}
}

\frame[containsverbatim]{
\frametitle{Prosta pętla}
Załóżmy, że chcemy zrealizować jakieś proste zadanie programistyczne.
Na przykład wykonać pewien kod w pętli.

Bierzemy nasz ulubiony asembler i kodujemy...

\begin{verbatim}
    mov bx, 10
    mov cx, 0
l1: 
    cmp cx, bx
    br_eq l2
    mov ax, cx
    inc
    mov cx, ax

    ...

    br l1
l2: 
\end{verbatim}
}

\frame[containsverbatim]{
\frametitle{Implementacja bardziej na czasie}
Ok, w asemblerze pisało się w moich czasach, teraz używamy C / C++ / C\#.

\begin{verbatim}
    int n = 10;
    int i = 0;
l1: 
    if (i == n)
      goto l2;
    i++;
    ...
    goto l1;
l2:
\end{verbatim}

Cały czas coś nie gra?
}

\frame[containsverbatim]{
\frametitle{Pierwszy ``design pattern'' w akcji}
Teraz lepiej. 
\begin{verbatim}
for (int i = 0; i < n; i++)
   // ...
\end{verbatim}

Wczujmy się w panów Kernighan i Ritchie'iego. Co zrobiliśmy? 
Właśnie wbudowaliśmy w nasz język pewną konstrukcję, która 
zawiera w sobie dość często stosowany schemat używany przez programistów - 
pętlę z inicjalizacją.
}

\frame{
\frametitle{Design pattern - co to?}
\emph{Design pattern} to ogólne rozwiązanie do pewnego 
  często spotykanego problemu w projektach programistycznych.

\begin{itemize}
  \item wykonanie czegoś wiele razy (pętla)
  \item modelowanie danych i relacji między nimi (struktury, klasy)
  \item schematy rozwiązań w projektowaniu obiektowym (to je powszechnie nazywa
    się \emph{design pattern})
  \item konwencje nazywania zmiennych, metod, klas i ich występowanie w
    określonych sytuacjach (getX(), setX() w Javie, właściwości w C\#)
  \item setki innych, o których ludzie piszą książki i zarabiają pieniądze
\end{itemize}

Jednym słowem - abstrakcja, uogólnianie i powszechne korzystanie ze sprawdzonych
pomysłów \footnote{ok, to już więcej niż jedno słowo}.
}

\section {Rozwój języków programowania}

\frame[containsverbatim] {
\frametitle{Iteratory w Javie}
\begin{verbatim}
  Iterator iter = coll.iterator ();
  while (iter.hasNext()) {
    Order ord = (Order) iter.next ();
    // ..
  }
\end{verbatim}

Pisze się je naprawdę dziesiątki, a nawet setki razy.
}

\frame[containsverbatim]{
\frametitle{foreach}
Trzeba było 8-miu lat, aby Sun wprowadził do języka skróconą formę składniową
lub jak kto woli, nowego języka (C\#), który wprowadził ją od samego początku, a
poza tym niewiele się od Javy różnił.

\begin{verbatim}
for (Widget w: box)
{
   System.out.println(w);
}
\end{verbatim}
}

\frame[containsverbatim] {
\frametitle{Nemerle - język oparty o rozszerzalność}
 Z ideami zawartymi w Nemerle skracamy ten wieloletni okres oczekiwania na
innowację do kilkunastu minut potrzebnych programiście, aby zaimplementował
następujące makro:

\begin{verbatim}
macro Foreach (n, coll, body)
syntax (``foreach'', ``(``, i, ``in'', coll, ``)'', body)
{
  <[ def iter = $coll.iterator ();
     while (iter.hasNext ()) {
       def $i = iter.next ();
       $body
     }
  ]>
}
\end{verbatim} % $
}

\section {Przykładowe schematy z programowania obiektowego}

\frame[containsverbatim]{
\frametitle{Adapter pattern}
Przystosowanie pewnego interfejsu do jakiegoś nieco innego, lecz
udostępniającego podobną funkcjonalność.
Mapuje pewien zestaw metod na inny zestaw metod.

$$Stack + ListImpl = StackListImpl$$

Tu możemy wykorzystać makra generujące metody, jak

\begin{verbatim}
<[ decl: 
   public Push (x : string) : int
   {
     // mapped into
     list.AddLast (x)
   } 
]>
\end{verbatim}
}

\frame{
\frametitle{Composite pattern}
Kontener na obiekty, które mają podobny zestaw metod. 
Udostępnia te metody i wywołuje je dla zawarych w nim obiektach.

$$Window + ListView + TreeView = Control$$
}

\section {Z życia wzięte}

\frame{
\frametitle{Jak wykorzystałbym to w firmie?}
\begin{itemize}
  \item generacja kodu
  \item generacja klas
  \item generacja pól w klasach
  \item generacja metod
  \item generacja kodu
\end{itemize}
}

\frame[containsverbatim]{
\frametitle{Przykład 1. - propagacja danych}
\begin{verbatim}
if (ord1.getReceiver() != null) 
  ord2.setReceiver (ord1.getReceiver());
if (ord1.getProducer() != null) 
  ord2.setProducer (ord1.getProducer());
if (ord1.getSpeditor() != null) 
  ord2.setSpeditor (ord1.getReceiver());
if (ord1.getAddress() != null) 
  ord2.setAddress (ord1.getAddress());
if (ord1.getIncoterms() != null) 
  ord2.setIncoterms (ord1.getIncoterms());
\end{verbatim}
}

\frame[containsverbatim]{
\frametitle{A makro potrafi generować...}
Możemy przecież pisać

\begin{verbatim}
propagateNotNull (ord1, ord2, 
                  Receiver, Producer, Speditor, 
                  Address, Incoterms);
\end{verbatim}

przy użyciu makra podobnego do 

\begin{verbatim}
macro propagateNotNull (o1, o2, names : list [PExpr]) {
   names.Map (fun (name) {
     <[ if ($o1.$name() != null) $o2.$name ($o1.$name ()) ]>
   })
}
\end{verbatim}
}

\frame[containsverbatim]{
\frametitle{Przykład 2. - propercje aplikacji}
\begin{verbatim}
class PropertyNames {
public static String PAGE_ORDER_RECEIVER_VISIBLE = 
  ``page.order.receiver.visible'';
  ...
}
class PropertyDefaults {
public static boolean PAGE_ORDER_RECEIVER_VISIBLE_DEFAULT = 
  false;
  ...
}
\end{verbatim}

a obok tego w bazie danych struktura drzewiasta

\begin{verbatim}
  ID  PARENTID KEY     TYPE   VALUE
  22  1        'order' 6      null
\end{verbatim}
}

\frame[containsverbatim]{
\frametitle{A jakbym chciał?}

\begin{verbatim}
[assembly: Properties (ReadFrom = DataBase, 
                       Source = 'woadm.companyptype')]
\end{verbatim}

\emph{Properties} jest makrem, które podczas kompilacji programu wczytuje drzewo
propercji z bazy danych (pliku XML, czegokolwiek) i wygenerujeć klasy 
\emph{PropertyNames} i \emph{PropertyDefaults} wypełniając je polami jak wyżej,
lub może zagnieżdżonymi klasami, żeby dopełnianie lepiej działało: 
\emph{Page.Order.Receiver.Visible}.
}

\frame[containsverbatim]{
\frametitle{Przykład 3. Klasy odpowiadające tabelom z bazy}
\begin{verbatim}
  class OrderPosition {
    Long id;
    Integer state;
    // ...

    public Long getId () { return id; }

    public OrderPositionData GetData () {
       OrderPositionData pd = new OrderPositionData ();
       pd.id = id;
       // ...
       return pd;
    }
  }
\end{verbatim}

i dodatkowo definicja \emph{OrderPositionData}...
}

\frame{
\frametitle{Po co klepać głupi kod?}
\begin{itemize}
\item Oczywiście cały ten kod, którego wklepanie, przetestowanie 
i utrzymanie kosztuje firmę cenne mendejsy, powinien zniknąć
i zostać wygenerowany automatycznie.

\item To samo tyczy się bardzo dużej części kodu nad którym pracują
firmy zajmujące się szarym biznesem.

\item Przyszłość leży w automatyzacji programowania, generacji kodu...
na szczęście ciągle ktoś musi pisać kod generujący kod.
\end{itemize}
}

\end{document}

% vim: language=polish
