\documentclass[a4paper,11pt]{article}
\usepackage[compat2,nofoot,headheight=14pt]{geometry}
\usepackage{fancyhdr}
\renewcommand{\ttdefault}{cmtt}
\renewcommand{\sfdefault}{pag}
\usepackage{url,array}

%\setlength{\parindent}{0pt}
%\setlength{\parskip}{5pt}

\overfullrule1pt

\hyphenation{De-bug-ger In-ter-pre-ter Spe-ci-fi-ca-tion}

\pagestyle{fancy}
\lhead{Nemerle project}
\rhead{\thepage}
\cfoot{}

\title{Nemerle project}
\author{Response to\\Second Rotor Request for Proposal}
\date{}

\begin{document}

\maketitle
\thispagestyle{empty}

\section{Overview}

\subsection{Project Title}

Nemerle -- project of new hybrid .NET language.

\subsection{Contributor Names}
Micha{\l} Moskal, Pawe{\l} Olszta, Kamil Skalski, Prof. Leszek Pacholski, Marcin M{\l}otkowski,
Tomasz Wierzbicki.

\subsection{Abstract}

The objective of this project is to design and implement a new 
hybrid (functional, object-oriented and imperative) programming 
language for the .NET platform. Key features of the language 
include: simplicity, a C\#-like syntax, easy to use object 
system (derived directly from the .NET), powerful code-generating 
macros, variants, pattern matching and type inference. We plan 
to make the language a full CLS consumer and producer, which 
fits the .NET as good as C\# does.

We will focus on investigating optimization opportunities 
raising as consequence of tightening the language's type system.
We will also try to investigate which way of implementing features
crucial to a functional language would yield the best performance 
on the SSCLI and the current .NET framework.


\section{Contact Details}


\subsection{Institution Details}

\begin{quote}
Institute of Computer Science\\
University of Wroc\l aw\\
ul.\ Przesmyckiego 20\\
PL--51--151 Wroc\l aw\\
Poland\\[2ex]
phone: 0048 71 3251271\\
fax:   0048 71 3756244\\
URL: \url{http://www.ii.uni.wroc.pl/}\\
URL: \url{http://nemerle.org/}
\end{quote}


\subsection{Contributor Details}
\begin{itemize}

\item
Micha{\l} Moskal (\texttt{michal.moskal@nemerle.org}), a master's
student and coordinator of the project. He was working on several large
programming projects, including few programming languages. URL: 
\url{http://www.kernel.pl/~malekith/}

\item
Pawe{\l} Olszta (\texttt{pawel.olszta@nemerle.org}), a master's student
and compiler developer. He has several years of industry experience, 
including 3 months spent in Redmond as an SDE intern. 

\item
Kamil Skalski (\texttt{kamil.skalski@nemerle.org}), a master' student.
He is responsible for the macro system design.

\item
Prof. Leszek Pacholski (\texttt{pacholsk@ii.uni.wroc.pl}), director of The Institute.

\item
Marcin M\l otkowski (\texttt{marcinm@ii.uni.wroc.pl}), lecturer at Wroc\l aw University.

\item
Tomasz Wierzbicki (\texttt{tomasz@ii.uni.wroc.pl}), assistant at Wroc\l aw University.

\end{itemize}

There are also several other students involved in the project.


\section{Project Description}

\subsection{Background}
Building programming language on the .NET framework base, makes large
variety of libraries for performing everyday programming tasks immediately
available. On the other hand lack of libraries (and in general problems
with implementation) seem to be one of the major stoppers for functional
programming languages. Designing language that will provide easy way
to access both its functional features and object oriented .NET world
would be major win.

The problem with ports of existing languages is retaining backward
compatibility -- it makes accessing .NET features harder.


\subsection{Project Details}

The primary goal of this project is to design and implement the language,
and later make people use it. The language project has already reached
mostly stable shape. Also the compiler is in stage where large class of
usable programs can be written. However the quality of compiler, and
especially documentation leaves much space for improvements.

Therefore the main objective of project submitted here is polishing the
implementation and documentation during next six months and maybe later.
This is detailed in Milestones section below.

The compiler currently uses boxing techniques to achieve parametric
polymorphism.  However the typesystem is largely modeled after .NET
Generics design \cite{generics}. We plan to start working on generic code
generator soon. As the desktop version of The Framework with generics
is not yet available, we are going to use Rotor with Gyro patches.


\subsection{Academic Relevance}

From the academic perspective, probably the most interesting part of
the language is our macro system. It extends to some extent
ideas of Meta-Haskell \cite{MetaHaskell}.

% collaboration with outside groups? MSR?
% cos jeszcze extendujemy, zebysmy mogli tu napisac?


\subsection{Experience}

Micha{\l} Moskal, the project coordinator has implemented few
toy-languages and three bigger ones. First one was Ksi -- front end for
GCC compiler, with lispy syntax, exposing internal syntax trees used in
GCC, thought as back end for other compilers.  Second one was Gont --
C-like language with parametric polymorphism, type inference, garbage
collection and higher order functions. The bootstraping Gont compiler
emits Ksi or C. The third one -- ET version 2 -- was interpreter for
language described by Zdzis{\l}aw Sp{\l}awski in his PhD dissertation.
It has strong normalization property, typesystem and syntax similar to ML,
semantics given by Lambda-2 translation, and has interesting connections
with Proof Theory.

% Ze mamy zaklad jezykow programowania, bla, bla.

\section{Project Plan}

\subsection{Deliverables/Milestones}
\begin{itemize}
\item version 0.1, scheduled Feb 15 2004. Features:
  mostly stable compiler and language;
  preliminary version of \textit{Reference Manual}; 
  macros working at the expression level.

\item version 0.2, scheduled Apr 15 2004. Features:
  most of CLS compatibility;
  macros on type definitions;
  documentation generation system;
  good pattern matching optimizations;
  some other simple optmizations;
  production version of \textit{Reference Manual};
  preliminary version of \textit{Nemerle for C\# Programmers} tutorial.
  
\item version 0.3, scheduled Jun 15 2004. Features:
  full CLS compatibility (producer/consumer);
  code generator using Generics;
  more compiler optimizations;
  production version of tutorial.
\end{itemize}

\subsection{Intellectual Property}

We're going to publish research papers in the area, however main outcome
is language implementation and documentation. Both will be released
under BSD-style license. See \url{http://nemerle.org/license.html}
for details.


\section{Supporting Information}


\subsection{Costing}

% Przelot do Redmond
% Jakies konferencje
%I sobie tez chcemy zaplacic


%\subsection{References}

\begin{thebibliography}{22}
\bibitem {generics}
Kennedy, A., Syme, D.:
Design and Implementation of Generics for the .NET Common Language Runtime.
Proceedings of PLDI, Jun. 2001.
\bibitem {Intro}
Moskal, M., Olszta, P., Skalski, K.:
Nemerle. Introduction to a Functional .NET Language.
Available from \url{http://nemerle.org/intro.pdf}.
\bibitem {MetaHaskell}
Sheard, T., Jones, S. P.:
Template Meta-programming for Haskell.
Haskell Workshop, Oct. 2002, Pittsburgh.
\end{thebibliography}


\end{document}

% ``letters of support'' from your institution or external groups ?
% 
